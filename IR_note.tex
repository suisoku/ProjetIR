\documentclass[]{article}
\usepackage{lmodern}
\usepackage{amssymb,amsmath}
\usepackage{ifxetex,ifluatex}
\usepackage{fixltx2e} % provides \textsubscript
\ifnum 0\ifxetex 1\fi\ifluatex 1\fi=0 % if pdftex
  \usepackage[T1]{fontenc}
  \usepackage[utf8]{inputenc}
\else % if luatex or xelatex
  \ifxetex
    \usepackage{mathspec}
  \else
    \usepackage{fontspec}
  \fi
  \defaultfontfeatures{Ligatures=TeX,Scale=MatchLowercase}
\fi
% use upquote if available, for straight quotes in verbatim environments
\IfFileExists{upquote.sty}{\usepackage{upquote}}{}
% use microtype if available
\IfFileExists{microtype.sty}{%
\usepackage{microtype}
\UseMicrotypeSet[protrusion]{basicmath} % disable protrusion for tt fonts
}{}
\usepackage[margin=1in]{geometry}
\usepackage{hyperref}
\hypersetup{unicode=true,
            pdftitle={Analyse des débats de l'assemblé national},
            pdfauthor={ZIANI,MASSELIN,DIALLO,ANZID},
            pdfborder={0 0 0},
            breaklinks=true}
\urlstyle{same}  % don't use monospace font for urls
\usepackage{color}
\usepackage{fancyvrb}
\newcommand{\VerbBar}{|}
\newcommand{\VERB}{\Verb[commandchars=\\\{\}]}
\DefineVerbatimEnvironment{Highlighting}{Verbatim}{commandchars=\\\{\}}
% Add ',fontsize=\small' for more characters per line
\usepackage{framed}
\definecolor{shadecolor}{RGB}{248,248,248}
\newenvironment{Shaded}{\begin{snugshade}}{\end{snugshade}}
\newcommand{\KeywordTok}[1]{\textcolor[rgb]{0.13,0.29,0.53}{\textbf{#1}}}
\newcommand{\DataTypeTok}[1]{\textcolor[rgb]{0.13,0.29,0.53}{#1}}
\newcommand{\DecValTok}[1]{\textcolor[rgb]{0.00,0.00,0.81}{#1}}
\newcommand{\BaseNTok}[1]{\textcolor[rgb]{0.00,0.00,0.81}{#1}}
\newcommand{\FloatTok}[1]{\textcolor[rgb]{0.00,0.00,0.81}{#1}}
\newcommand{\ConstantTok}[1]{\textcolor[rgb]{0.00,0.00,0.00}{#1}}
\newcommand{\CharTok}[1]{\textcolor[rgb]{0.31,0.60,0.02}{#1}}
\newcommand{\SpecialCharTok}[1]{\textcolor[rgb]{0.00,0.00,0.00}{#1}}
\newcommand{\StringTok}[1]{\textcolor[rgb]{0.31,0.60,0.02}{#1}}
\newcommand{\VerbatimStringTok}[1]{\textcolor[rgb]{0.31,0.60,0.02}{#1}}
\newcommand{\SpecialStringTok}[1]{\textcolor[rgb]{0.31,0.60,0.02}{#1}}
\newcommand{\ImportTok}[1]{#1}
\newcommand{\CommentTok}[1]{\textcolor[rgb]{0.56,0.35,0.01}{\textit{#1}}}
\newcommand{\DocumentationTok}[1]{\textcolor[rgb]{0.56,0.35,0.01}{\textbf{\textit{#1}}}}
\newcommand{\AnnotationTok}[1]{\textcolor[rgb]{0.56,0.35,0.01}{\textbf{\textit{#1}}}}
\newcommand{\CommentVarTok}[1]{\textcolor[rgb]{0.56,0.35,0.01}{\textbf{\textit{#1}}}}
\newcommand{\OtherTok}[1]{\textcolor[rgb]{0.56,0.35,0.01}{#1}}
\newcommand{\FunctionTok}[1]{\textcolor[rgb]{0.00,0.00,0.00}{#1}}
\newcommand{\VariableTok}[1]{\textcolor[rgb]{0.00,0.00,0.00}{#1}}
\newcommand{\ControlFlowTok}[1]{\textcolor[rgb]{0.13,0.29,0.53}{\textbf{#1}}}
\newcommand{\OperatorTok}[1]{\textcolor[rgb]{0.81,0.36,0.00}{\textbf{#1}}}
\newcommand{\BuiltInTok}[1]{#1}
\newcommand{\ExtensionTok}[1]{#1}
\newcommand{\PreprocessorTok}[1]{\textcolor[rgb]{0.56,0.35,0.01}{\textit{#1}}}
\newcommand{\AttributeTok}[1]{\textcolor[rgb]{0.77,0.63,0.00}{#1}}
\newcommand{\RegionMarkerTok}[1]{#1}
\newcommand{\InformationTok}[1]{\textcolor[rgb]{0.56,0.35,0.01}{\textbf{\textit{#1}}}}
\newcommand{\WarningTok}[1]{\textcolor[rgb]{0.56,0.35,0.01}{\textbf{\textit{#1}}}}
\newcommand{\AlertTok}[1]{\textcolor[rgb]{0.94,0.16,0.16}{#1}}
\newcommand{\ErrorTok}[1]{\textcolor[rgb]{0.64,0.00,0.00}{\textbf{#1}}}
\newcommand{\NormalTok}[1]{#1}
\usepackage{graphicx,grffile}
\makeatletter
\def\maxwidth{\ifdim\Gin@nat@width>\linewidth\linewidth\else\Gin@nat@width\fi}
\def\maxheight{\ifdim\Gin@nat@height>\textheight\textheight\else\Gin@nat@height\fi}
\makeatother
% Scale images if necessary, so that they will not overflow the page
% margins by default, and it is still possible to overwrite the defaults
% using explicit options in \includegraphics[width, height, ...]{}
\setkeys{Gin}{width=\maxwidth,height=\maxheight,keepaspectratio}
\IfFileExists{parskip.sty}{%
\usepackage{parskip}
}{% else
\setlength{\parindent}{0pt}
\setlength{\parskip}{6pt plus 2pt minus 1pt}
}
\setlength{\emergencystretch}{3em}  % prevent overfull lines
\providecommand{\tightlist}{%
  \setlength{\itemsep}{0pt}\setlength{\parskip}{0pt}}
\setcounter{secnumdepth}{0}
% Redefines (sub)paragraphs to behave more like sections
\ifx\paragraph\undefined\else
\let\oldparagraph\paragraph
\renewcommand{\paragraph}[1]{\oldparagraph{#1}\mbox{}}
\fi
\ifx\subparagraph\undefined\else
\let\oldsubparagraph\subparagraph
\renewcommand{\subparagraph}[1]{\oldsubparagraph{#1}\mbox{}}
\fi

%%% Use protect on footnotes to avoid problems with footnotes in titles
\let\rmarkdownfootnote\footnote%
\def\footnote{\protect\rmarkdownfootnote}

%%% Change title format to be more compact
\usepackage{titling}

% Create subtitle command for use in maketitle
\providecommand{\subtitle}[1]{
  \posttitle{
    \begin{center}\large#1\end{center}
    }
}

\setlength{\droptitle}{-2em}

  \title{Analyse des débats de l'assemblé national}
    \pretitle{\vspace{\droptitle}\centering\huge}
  \posttitle{\par}
    \author{ZIANI,MASSELIN,DIALLO,ANZID}
    \preauthor{\centering\large\emph}
  \postauthor{\par}
      \predate{\centering\large\emph}
  \postdate{\par}
    \date{09 mai 2019}


\begin{document}
\maketitle

\subsection{\texorpdfstring{\emph{Initiation à la
Recherche}}{Initiation à la Recherche}}\label{initiation-a-la-recherche}

Encadré par : LABBE Cyril

Réalisé par : DIALLO Thierno MASSELIN Thibaut ZIANI Nour-eddine ANZID
Samya

Année universitaire 2018/2019

\section{Résumé}\label{resume}

Présentation d'une analyse réalisée sur un débat de l'Assemblée
nationale. On traite les données brutes récupérer sur le site officiel
des allocutions des participants. Ce type de donnée se prête bien à
l'analyse textuelle, pour mettre en évidence certains sentiments en les
classifications par famille. La méthode offre la possibilité de
qualifier l'état d'esprit d'un texte.

\section{Introduction}\label{introduction}

Notre sujet porte sur les documents de l'Assemblée nationale français.
On s'est intéressé à pouvoir déterminer dans quel état d'esprit se
déroule un débat à l'Assemblée nationale. L'article est organisé en
différentes parties une brève explication de la méthodologie. La partie
suivante permettra de présenter les résultats obtenus, puis nous
évoquerons notre analyse à partir des précédées données et une brève
conclusion.

\subsection{Démarche d'acquisition des données et construction des
données}\label{demarche-dacquisition-des-donnees-et-construction-des-donnees}

Pour collecter les données nécessaire à réaliser notre étude, nous
sommes allés les récupérer directement depuis le site de l'assemblé
national. Les données sont de très bonne qualité car ce sont des
professionnels qui réalisent ces transcriptions. Nous avons sélectionné
un débat qui c'est déroulée pendant la 123eme séance du 12 février 2017.
Pour ce faire, nous avons dû utiliser la librairie ``rvest'' sur rstudio
qui permet d'importer l'arbre html de la page. Puis dans un second temp,
on avait extrait les textes de chaque allocution chronologiquement.
Enfin pour enrichir les données extraites, nous avons discriminer les
données en fonction du sexe et du rôle exercé par locution. Puis dans un
second temps nous avons isolé chaque mots via une fonction de R tout en
conserve les informations de son énonciateur. Pour augmenter la
pertinence pendant l'analyse sur la liste de mot, nous avons fait le
choix de retirer les mots qui ne sont pas significatif (mots non porteur
de sens). Par la suite on a continuer à diviser les tâches entre nous
sur l'analyse des sentiments, les bigrams par fréquence et les mots les
plus utilisés.

\subsection{Question sur laquelle se base
l'analyse}\label{question-sur-laquelle-se-base-lanalyse}

\textbf{Dans quelle atmosphère se déroulent les débats au sein de
l'Assemblée nationale par l'analyse textuelle.}

\subsection{Méthodes de travail}\label{methodes-de-travail}

Dans une approche de travail en groupe (4 personnes), nous avons décidé
d'utiliser différents outils de collaboration et de partage de source.
Le principal outil utilisé a été un ``Repositories Git'' qui permet de
gérer les différents travaux en concurrence et garder des versions sur
serveur. Pour débuter, nous avons choisi de faire de la veille sur les
différents techniques existant par la lecture de publication
scientifique en rapport avec notre problématique. Notre tuteur nous a
envoyé de nombreux document qui nous a beaucoup aidé sur la
compréhension du sujet et savoir bien ce qui demander à faire. Puis dans
un second temps, nous utiliserons l'IDE R-studio pour récupérer nos jeux
donnés et réaliser des analyses par le moyen du graphique ou de résultat
numérique.

\section{Analyse Statistique (Text
mining)}\label{analyse-statistique-text-mining}

\subsection{Chargement des Packages
utilisées}\label{chargement-des-packages-utilisees}

\begin{Shaded}
\begin{Highlighting}[]
\KeywordTok{install.packages}\NormalTok{(}\StringTok{"syuzhet"}\NormalTok{)}
\end{Highlighting}
\end{Shaded}

\begin{verbatim}
## Installing package into '/home/masselit/R/x86_64-pc-linux-gnu-library/3.4'
## (as 'lib' is unspecified)
\end{verbatim}

\begin{Shaded}
\begin{Highlighting}[]
\KeywordTok{install.packages}\NormalTok{(}\StringTok{"tm"}\NormalTok{)}
\end{Highlighting}
\end{Shaded}

\begin{verbatim}
## Installing package into '/home/masselit/R/x86_64-pc-linux-gnu-library/3.4'
## (as 'lib' is unspecified)
\end{verbatim}

\begin{Shaded}
\begin{Highlighting}[]
\KeywordTok{install.packages}\NormalTok{(}\StringTok{"rvest"}\NormalTok{)}
\end{Highlighting}
\end{Shaded}

\begin{verbatim}
## Installing package into '/home/masselit/R/x86_64-pc-linux-gnu-library/3.4'
## (as 'lib' is unspecified)
\end{verbatim}

\begin{Shaded}
\begin{Highlighting}[]
\KeywordTok{install.packages}\NormalTok{(}\StringTok{"xml2"}\NormalTok{)}
\end{Highlighting}
\end{Shaded}

\begin{verbatim}
## Installing package into '/home/masselit/R/x86_64-pc-linux-gnu-library/3.4'
## (as 'lib' is unspecified)
\end{verbatim}

\begin{Shaded}
\begin{Highlighting}[]
\KeywordTok{install.packages}\NormalTok{(}\StringTok{"stringi"}\NormalTok{)}
\end{Highlighting}
\end{Shaded}

\begin{verbatim}
## Installing package into '/home/masselit/R/x86_64-pc-linux-gnu-library/3.4'
## (as 'lib' is unspecified)
\end{verbatim}

\begin{Shaded}
\begin{Highlighting}[]
\KeywordTok{install.packages}\NormalTok{(}\StringTok{"stringr"}\NormalTok{)}
\end{Highlighting}
\end{Shaded}

\begin{verbatim}
## Installing package into '/home/masselit/R/x86_64-pc-linux-gnu-library/3.4'
## (as 'lib' is unspecified)
\end{verbatim}

\begin{Shaded}
\begin{Highlighting}[]
\KeywordTok{install.packages}\NormalTok{(}\StringTok{"dplyr"}\NormalTok{)}
\end{Highlighting}
\end{Shaded}

\begin{verbatim}
## Installing package into '/home/masselit/R/x86_64-pc-linux-gnu-library/3.4'
## (as 'lib' is unspecified)
\end{verbatim}

\begin{Shaded}
\begin{Highlighting}[]
\KeywordTok{install.packages}\NormalTok{(}\StringTok{"tidytext"}\NormalTok{)}
\end{Highlighting}
\end{Shaded}

\begin{verbatim}
## Installing package into '/home/masselit/R/x86_64-pc-linux-gnu-library/3.4'
## (as 'lib' is unspecified)
\end{verbatim}

\begin{Shaded}
\begin{Highlighting}[]
\KeywordTok{install.packages}\NormalTok{(}\StringTok{"ggplot2"}\NormalTok{)}
\end{Highlighting}
\end{Shaded}

\begin{verbatim}
## Installing package into '/home/masselit/R/x86_64-pc-linux-gnu-library/3.4'
## (as 'lib' is unspecified)
\end{verbatim}

\subsection{L'analyse des sentiments}\label{lanalyse-des-sentiments}

Suite au traitement décrit au préalable dans la partie démarche et
acquisition données. À partir des textes extraits, nous réalisons
l'analyse des sentiments pour tenter de prévoir les réactions, les
attitudes, le contexte et les émotions des députés. On se concentre
notamment sur la différence entre les hommes et les femmes. Cette
analyse est basé sur un ensemble de mots `Lexiques' qu'ils sont divisés
en trois parties positifs, négatifs et neutre. Pour cela nous avons
utilisé le ``package syuzhet''. En revanche, cette analyse s'effectue
par l'intersection des mots extraient dans la première partie et les
mots du package utiliser.

\begin{Shaded}
\begin{Highlighting}[]
\KeywordTok{library}\NormalTok{(rvest)}
\end{Highlighting}
\end{Shaded}

\begin{verbatim}
## Loading required package: xml2
\end{verbatim}

\begin{Shaded}
\begin{Highlighting}[]
\KeywordTok{library}\NormalTok{(xml2)}
\KeywordTok{library}\NormalTok{(stringi)}
\CommentTok{# Extraction de la page html d'une séance unique (123eme) du 22/02/2017}
\NormalTok{textTreeBase <-}\StringTok{ }\KeywordTok{read_html}\NormalTok{(}\StringTok{"http://www.assemblee-nationale.fr/14/cri/2016-2017/20170123.asp"}\NormalTok{)}

\CommentTok{# Extraction }
\NormalTok{textSeance20170123<-}\StringTok{ }\NormalTok{textTreeBase }\OperatorTok\StringTok{ }\KeywordTok{html_nodes}\NormalTok{(}\StringTok{"body"}\NormalTok{) }\OperatorTok\StringTok{ }\KeywordTok{html_nodes}\NormalTok{(}\StringTok{"p"}\NormalTok{) }\OperatorTok\StringTok{ }\KeywordTok{html_text}\NormalTok{() }\CommentTok{#extraction dans le noeud <body> puis dnas le noeud <p>}
\end{Highlighting}
\end{Shaded}

On Observe que le premier filtre n'est pas adéquate car il retire bien
trop de mot sigificatif comme ``politiques''

Donc dans la suite des traitement nous utiliserons que le filtre 2 si
besoin.

Pour ces deux graphes, on a analysé les termes utiliser dans les débats
sans differencier entre homme et femme.

Ce graphe nous permet de deduire que il utiliser les mots de confiance
le plus souvent apres on voit les termes d'anticipation et de peur.

Ce graphe permet d'illustrer que les termes positifs domine sur les
termes negatifs dans les débats.

En comparant le premier graphe entre femme et homme on peut deduire que
les hommes parlent plus avec des termes de sentiment d'anticipation et
de confiance alors que les femmes parlent de sentiment de peur,
tristesse.La on peut deduire que les femmes y a une grande partie dans
leur discours des expressions d'?motions plus que les hommes.

\subsubsection{Methodologie et Analyse}\label{methodologie-et-analyse}

Il peut etre interessant de montrer les relatations ou des correleations
entre les mots pour cela nous prodedons au calculs bi-grams : sequence
continue de 2 mots

D'abord nous allons sectionner le tableau de mot duc corpus en tokens de
2 Puis on genere un tableau avec 2 colonnes reprenants les token
calcules Puis on enleve les mots non pertinants Finallement on trie les
bigrams par fequence

Resultat : Les bigrams comme secretaire d'etat et groupe socialiste
apparaissent le plus soucant dans ce corpus ce qui surligne leur
importance dans le debat.


\end{document}
